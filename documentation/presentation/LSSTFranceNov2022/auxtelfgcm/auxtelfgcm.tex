% Copyright 2022 by Sébastien Descotes-Genon
%
% This file may be distributed and/or modified
%
% 1. under the LaTeX Project Public License and/or
% 2. under the GNU Public License.
%


\documentclass{beamer}

% Setup appearance:

\usetheme{IJCLab}

\pgfdeclareimage[height=2cm]{LogoIJCLab}{example-fig/LogoIJCLab}
\pgfdeclareimage[height=1.3cm]{LogoCNRS}{example-fig/LogoCNRS}
\pgfdeclareimage[height=1cm]{LogoUPSay}{example-fig/LogoUPSay}


\usepackage[english]{babel}
\usepackage[T1]{fontenc}


% Author, Title, etc.

\title[FGCM]{Forward Global Photometric Calibration at Rubin-LSST with Auxtel}

\titlegraphic{\begin{center}\qquad\pgfuseimage{LogoCNRS}\qquad\quad\qquad\quad\pgfuseimage{LogoIJCLab}\qquad\qquad\pgfuseimage{LogoUPSay}\end{center}}

\author[S. Dagoret-Campagne]{
Sylvie Dagoret-Campagne}
\institute[IJCLab]{
  IJCLab,
  CNRS/IN2P3 \& Université Paris-Saclay,
  Orsay, France   }
  
  % Caution  : This is not the affiliation to be used to sign articles. 
  % Please refer to the intranet : Bibliothèque/IST / Règles de publication
     
\date[Orsay, November 2022]{Photometric corrections with atmospheric calibration}


% The main document with a few examples

\begin{document}

\begin{frame}
  \titlepage
\end{frame}

\begin{frame}{Outline}
  \tableofcontents
\end{frame}

\section{Definitions for FGCM}
\subsection{Instrumental Flux}

%==================================================================================================================================
\begin{frame}{Instrumental Flux}
\begin{alertblock}{Photometric flux of a source}
	\begin{equation}
	ADU_b = \frac{A\Delta T}{gh}\int_0^\infty F_\nu(\lambda) \times S_b^{obs}(\lambda,x,y,az,alt,t) \frac{d\lambda}{\lambda}
	\end{equation}
	\begin{itemize}
	\item $ADU_b$ : ADU count of a source measured by photometry in band $b$
	\item $F_\nu$ : SED in ${\rm erg/cm^2/Hz/s}$
	\item $S_b^{obs}$ : Observation transmission in band $b$(atmosphere + instrument)
	\item $A$ : Collection efficiency in ${\rm cm^2}$
	\item $g$ : Electronic gain in ${\rm e^-/ADU}$
	\item $\Delta T$ : Exposure time
	\item $h$ : Planck constant
	\end{itemize}
	\end{alertblock}	
\end{frame}
%=================================================================================================================================


\subsection{Observed Magnitude}
%=================================================================================================================================
\begin{frame}{Observed Magnitude}
\begin{alertblock}{Observed Magnitude of a source}
	\begin{equation}
	m^{obs}_b = -2.5 \log_{10}\left( 
	\frac{\int_0^\infty F_\nu(\lambda) \times S_b^{obs}(\lambda,x,y,az,alt,t) \frac{d\lambda}{\lambda} }{\int_0^\infty F^{AB} \times S_b^{obs}(\lambda,x,y,az,alt,t) \frac{d\lambda}{\lambda}} 
	\right)
	\end{equation}
	\begin{itemize}
	\item $m^{obs}_b$ : observed magnitude in band $b$
	\item $F_\nu$ : SED in ${\rm erg/cm^2/Hz/s}$
	\item $F^{AB}=3631 Jy$ : Flat SED with $1 Jy = 10^{-23} {\rm erg/cm^2/Hz/s}$
	\item $S_b^{obs}$ : Observation transmission in band $b$  (atmosphere + instrument)
	\end{itemize}
	\end{alertblock}	
In the above formula gives how the physics provides $m_b^{obs}$. However usually the $F_\nu(\lambda)$ of an object is unknown.	
Note $m^{obs}_b$ is defined independently of any reference to a standard magnitude.
\end{frame}
%========================================================================================================================



%========================================================================================================================
\begin{frame}{Natural magnitude in Rubin LSST}
Rubin-LSST usually use the concept of normalized passband
\begin{alertblock}{normalized bandpass response function}
\begin{equation}
\phi_b^{obs} (\lambda,t) = \frac{S_b^{obs}(\lambda,t)\frac{1}{\lambda}}{\int_0^\infty S^{obs}_b(\lambda,t) \frac{d\lambda}{\lambda}}
\end{equation}
\end{alertblock}
\begin{block}{Observed flux}
\begin{equation}
F_b^{obs} = \int_0^\infty F_\nu(\lambda) \phi_b^{obs}(\lambda) d\lambda
\end{equation}
\end{block}
\begin{exampleblock}{Natural magnitude}
\begin{equation}
m_b^{nat} = -2.5 \log_{10} \left( \frac{F_b^{obs}}{F^{AB}}\right)
\end{equation}
\end{exampleblock}
The natural magnitude $m_b^{nat}$ is similar to  the observed magnitude $m_b^{obs}$
\end{frame}
%=========================================================================================================================



%========================================================================================================================
\begin{frame}{Decomposition of observed Magnitude} 
The observed magnitude is estimated from the measured ADU counts rate $C_b=ADU_b/\Delta T$ in filter $b$
\begin{alertblock}{}
\begin{equation}
m^{obs}_b = -2.5 \log_{10}(C_b)+ 2.5 \log_{10}\left(\mathbb{I}_0^{obs}(b)\right) + ZPT^{AB}
\end{equation}	
\end{alertblock}	
However the two following quantities which depend on atmospheric + detectors conditions must be estimated by calibration. 
\begin{block}{Calibration quantities:}
\begin{eqnarray}
\mathbb{I}_0^{obs}(b) & \equiv & \int_0^\infty S^{obs}_b(\lambda) \frac{d\lambda}{\lambda} \\
ZPT^{AB} & \equiv & 2.5 \log_{10} \left( \frac{A F^{AB}}{gh}\right)
\end{eqnarray} 
\end{block}

\end{frame}
%==========================================================================================================================

%==========================================================================================================================
\begin{frame}{Zero point} 
Note sometimes one defines the zero point as the magnitude $m_b^{obs}$, such $ADU_b/\Delta T = 1$ count per sec, then
\begin{equation}
m^{obs}_b(ZP) \equiv 2.5\log_{10}\left( \mathbb{I}_0^{obs}(b)\right) + ZPT^{AB}
\end{equation}	
\begin{itemize}
\item the zero point is common to all sources whatever their color is 
\item it has a time  dependent component : $2.5\log_{10}\left( \mathbb{I}_0^{obs}(b)\right)$
\item it has a detector (CCD) dependent component :  $ZPT^{AB}$ through the relative electronic gain $g$
\end{itemize}
\end{frame}
%==========================================================================================================================



\subsection{Standard Magnitude}
%==========================================================================================================================
\begin{frame}{Standard Magnitude} 
Standard magnitude is the magnitude to be published with the standard passband. 
It must be calculated from the observed magnitude.
\begin{alertblock}{Standard magnitude in standard passband}
	\begin{equation}
	m^{std}_b = -2.5 \log_{10}
	\left( 
	\frac{\int_0^\infty F_\nu(\lambda) \times S_b^{std}(\lambda) \frac{d\lambda}{\lambda} }{\int_0^\infty F^{AB} \times S_b^{std}(\lambda) \frac{d\lambda}{\lambda}} 
	\right)
	\end{equation}	
\end{alertblock}	

\begin{eqnarray}
\delta^{std}_b & \equiv & m_b^{std} - m_b^{obs} \\
& \equiv & 2.5 \log_{10}\left( \frac{\mathbb{I}_0^{std}(b)}{\mathbb{I}_0^{obs}(b)}\right) 
+ 2.5 \log_{10} 
	\left( 
	\frac{\int_0^\infty F_\nu(\lambda) \times S_b^{obs}(\lambda) \frac{d\lambda}{\lambda} }{\int_0^\infty F_\nu(\lambda) \times S_b^{std}(\lambda) \frac{d\lambda}{\lambda}} 
	\right)
\end{eqnarray} 
\end{frame}
%==========================================================================================================================

%==========================================================================================================================
\begin{frame}{Standard magnitude in Rubin LSST}
\begin{alertblock}{standard magnitude}
\begin{eqnarray}
m_b^{std} & = & m_b^{nat} + \Delta m_b^{obs} \\
\Delta m_b^{obs} & = & 2.5 \log_{10} \frac{\int_0^\infty F_\nu(\lambda) \phi_b^{obs}(\lambda) d\lambda}{\int_0^\infty F_\nu(\lambda) \phi_b^{std}(\lambda) d\lambda}
\end{eqnarray}
\end{alertblock}
\begin{block}{}
\begin{equation}
\Delta m_b^{obs} = \delta_b^{std} =  2.5 \log_{10}\left( \frac{\mathbb{I}_0^{std}(b)}{\mathbb{I}_0^{obs}(b)}\right) 
+ 2.5 \log_{10} 
	\left( 
	\frac{\int_0^\infty F_\nu(\lambda) \times S_b^{obs}(\lambda) \frac{d\lambda}{\lambda} }{\int_0^\infty F_\nu(\lambda) \times S_b^{std}(\lambda) \frac{d\lambda}{\lambda}} 
	\right)
\end{equation}
\end{block}
\end{frame}
%==========================================================================================================================




%==========================================================================================================================
\begin{frame}{Zero Point in Rubin-LSST}
\begin{exampleblock}{Zero point definition}
\begin{eqnarray}
m_b^{std} & = & -2.5 \log_{10}(C_b^{obs}) + \Delta m_b^{obs} + Z_b^{obs}
\end{eqnarray}
with $C_b^{obs} = ADU_b/\Delta T$
\end{exampleblock}
\begin{alertblock}{Correspondence of Zero point in Rubin-DES}
\begin{equation}
Z_b^{obs} = 2.5\log_{10}\left(\mathbb{I}_0^{obs}(b)\right) + ZPT^{AB}
\end{equation}
\end{alertblock}
\end{frame}
%==========================================================================================================================


%==========================================================================================================================
\begin{frame}{Interpretable standard magnitude expression}
\begin{alertblock}{option A : with zero point as unit counting rate}
\begin{eqnarray}
m_b^{std} & = &  -2.5 \log_{10}(C_b^{obs})+ 2.5 \log_{10}\left(\frac{\mathbb{I}_0^{std}(b)}{\mathbb{I}_0^{obs}}\right) + m_b^{obs}(ZPT)   \nonumber \\
 & & + 2.5 \log_{10} 
	\left( 
	\frac{\int_0^\infty F_\nu(\lambda) \times S_b^{obs}(\lambda) \frac{d\lambda}{\lambda} }{\int_0^\infty F_\nu(\lambda) \times S_b^{std}(\lambda) \frac{d\lambda}{\lambda}} 
	\right)
\end{eqnarray}
\end{alertblock}
\begin{itemize}
\item $-2.5 \log_{10}(C_b^{obs})$ : measured aperture photometric term
\item $2.5 \log_{10}\left(\frac{\mathbb{I}_0^{std}(b)}{\mathbb{I}_0^{obs}}\right)$ : SED color free correction term for atmospheric transparency standard/observed
\item $m_b^{obs}(ZPT)$ :  SED color free observed magnitude Zero Point correction term at CCD level,
\item $2.5 \log_{10} 
	\left( 
	\frac{\int_0^\infty F_\nu(\lambda) \times S_b^{obs}(\lambda) \frac{d\lambda}{\lambda} }{\int_0^\infty F_\nu(\lambda) \times S_b^{std}(\lambda) \frac{d\lambda}{\lambda}}\right)$ : SED color dependent term correction
\end{itemize}
\end{frame}
%============================================================================================




%==========================================================================================================================
\begin{frame}{Interpretable standard magnitude expression}
\begin{alertblock}{option B : with zero point as constants}
\begin{eqnarray}
m_b^{std} & = &  -2.5 \log_{10}(C_b^{obs})+ 2.5 \log_{10}\left(\mathbb{I}_0^{std}(b)\right) + ZPT^{AB}   \nonumber \\
 & & + 2.5 \log_{10} 
	\left( 
	\frac{\int_0^\infty F_\nu(\lambda) \times S_b^{obs}(\lambda) \frac{d\lambda}{\lambda} }{\int_0^\infty F_\nu(\lambda) \times S_b^{std}(\lambda) \frac{d\lambda}{\lambda}} 
	\right)
\end{eqnarray}
\end{alertblock}
\begin{itemize}
\item $-2.5 \log_{10}(C_b^{obs})$ : measured aperture photometric term
\item $2.5 \log_{10}\left(\mathbb{I}_0^{std}(b)\right)$ : Calculable constant
\item $ZPT^{AB} = \frac{A F^{AB}}{gh}$ :  SED color free observed magnitude Zero Point correction term at CCD level (electronic gain),
\item $2.5 \log_{10} 
	\left( 
	\frac{\int_0^\infty F_\nu(\lambda) \times S_b^{obs}(\lambda) \frac{d\lambda}{\lambda} }{\int_0^\infty F_\nu(\lambda) \times S_b^{std}(\lambda) \frac{d\lambda}{\lambda}}\right)$ : SED color dependent term correction
\end{itemize}
\end{frame}
%============================================================================================




\subsection{Taylor Development of the SED}
%==========================================================================================================================
\begin{frame}{Taylor Development of the SED} 
\begin{exampleblock}{SED approximation}
\begin{eqnarray}
F_\nu(\lambda) & = & F_\nu(\lambda_b) \left(1 + f^\prime(\lambda_b)(\lambda-\lambda_b) + \frac{f^{\prime\prime}(\lambda_b)}{2}(\lambda-\lambda_b)^2 + \cdots \right) \\
f_\nu^\prime(\lambda) & \equiv & \frac{1}{F_\nu(\lambda)}\frac{dF_\nu(\lambda)}{d\lambda} \qquad \qquad \qquad
f_\nu^{\prime\prime}(\lambda)  \equiv  \frac{1}{F_\nu(\lambda)}\frac{d^2F_\nu(\lambda)}{d\lambda^2} \nonumber \\
\lambda_b & \equiv & \frac{\int_0^{\infty} \lambda \times S_b^{inst}(\lambda) \frac{d\lambda}{\lambda}}
{\int_0^{\infty} S_b^{inst}(\lambda) \frac{d\lambda}{\lambda}}
\end{eqnarray}
\begin{itemize}
\item $S_b^{obs}(\lambda) = S_b^{inst}(\lambda,t,x,y) \times S^{atm}(\lambda,t,alt,az)$
\end{itemize}
\end{exampleblock}
\end{frame}
%==========================================================================================================================

\subsection{Summary of definitions on interpretable standard magnitude}
%==========================================================================================================================
\begin{frame}{Summary of definitions}
\begin{alertblock}{Decomposition of standard magnitude}
\begin{eqnarray}
m_b^{std} & = & -2.5 \log_{10}(C_b)  \nonumber \\
          &   &  + 2.5 \log_{10}(\mathbb{I}_0^{obs}(b)) + ZPT^{AB}  \nonumber \\
          &   &  + 2.5 \log_{10}\left( 
\frac{1 + f_\nu^\prime(\lambda_b)\mathbb{I}^{obs}_{10}(b)  +\frac{f_\nu^{\prime\prime}(\lambda_b)}{2}\mathbb{I}_{20}^{obs}(b)}
{1 + f_\nu^\prime(\lambda_b)\mathbb{I}^{std}_{10}(b)  +\frac{f_\nu^{\prime\prime}(\lambda_b)}{2}\mathbb{I}_{20}^{std}(b)}\right)
\end{eqnarray}
\end{alertblock}

\begin{block}{Integral definition}
{\footnotesize
$
\begin{array}{lll}
\mathbb{I}_0^i(b) = \int_0^{\infty} S_b^i(\lambda) \frac{d\lambda}{\lambda}
& \mathbb{I}_1^i(b) = \int_0^{\infty} (\lambda-\lambda_b) S_b^i(\lambda) \frac{d\lambda}{\lambda}
& \mathbb{I}_2^i(b) = \int_0^{\infty} (\lambda-\lambda_b)^2 S_b^i(\lambda) \frac{d\lambda}{\lambda} \\
\mathbb{I}^i_{10}(b) = \frac{\mathbb{I}^i_1(b)}{\mathbb{I}^i_0(b)} &
\mathbb{I}^i_{20}(b) = \frac{\mathbb{I}^i_2(b)}{\mathbb{I}^i_0(b)} & 
\end{array}
$
}
with $i=obs$ or $i=std$
\end{block}
\end{frame}
%==========================================================================================================================


%==========================================================================================================================
\begin{frame}{Approximation of color correction}
\begin{alertblock}{Decomposition of standard magnitude}
\begin{eqnarray}
m_b^{std} & = & -2.5 \log_{10}(C_b)  \nonumber \\
          &   &  + 2.5 \log_{10}(\mathbb{I}_0^{obs}(b)) + ZPT^{AB}  \nonumber \\
          &   &  + 1.087\left( f_\nu^\prime(\lambda_b) \Delta \mathbb{I}_{10}(b) +
          \frac{f_\nu^{\prime\prime}(\lambda_b)}{2}\Delta \mathbb{I}_{20}(b) \right. \nonumber \\
         & & - \left. \frac{1}{2}\left( f_\nu^\prime(\lambda_b) \Delta \mathbb{I}_{10}(b) \right)^2             
          \right)
\end{eqnarray}
\end{alertblock}
\begin{block}{Integral definition}
\begin{eqnarray}
\Delta \mathbb{I}_{10}(b) & = &  \mathbb{I}_{10}^{obs}(b)  -  \mathbb{I}_{10}^{std}(b) \\
\Delta \mathbb{I}_{20}(b) & = &   \mathbb{I}_{20}^{obs}(b)  -  \mathbb{I}_{20}^{std}(b) 
\end{eqnarray}
\end{block}
\end{frame}
%==========================================================================================================================


\begin{frame}
\begin{center}
{\usebeamerfont{frametitle}

\LARGE \alert{Thanks for your attention}}

\end{center}

\end{frame}


 
\end{document}


