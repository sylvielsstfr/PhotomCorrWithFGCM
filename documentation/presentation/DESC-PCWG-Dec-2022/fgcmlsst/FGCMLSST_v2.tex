% Copyright 2022 by Sébastien Descotes-Genon
%
% This file may be distributed and/or modified
%
% 1. under the LaTeX Project Public License and/or
% 2. under the GNU Public License.
%


\documentclass{beamer}

\usepackage{array,amsmath}
\newcolumntype{L}{>$l<$}
\usepackage{hyperref}


% Setup appearance:

\usetheme{IJCLab}

\pgfdeclareimage[height=2cm]{LogoIJCLab}{example-fig/LogoIJCLab}
\pgfdeclareimage[height=1.3cm]{LogoCNRS}{example-fig/LogoCNRS}
\pgfdeclareimage[height=1cm]{LogoUPSay}{example-fig/LogoUPSay}


\usepackage[english]{babel}
\usepackage[T1]{fontenc}


%\AtBeginSection[]
%{
%  \begin{frame}
%    \frametitle{Table of Contents}
%    \tableofcontents[currentsection]
%  \end{frame}
%}



% Author, Title, etc.

\title[FGCM Simulation]{\large Review of the Forward Global Photometric Calibrationn\\
{\small inspired from the DES article Burke et al. 2017}}

\titlegraphic{\begin{center}\qquad\pgfuseimage{LogoCNRS}\qquad\quad\qquad\quad\pgfuseimage{LogoIJCLab}\qquad\qquad\pgfuseimage{LogoUPSay}\end{center}}

\author[S. Dagoret-Campagne]{
Sylvie Dagoret-Campagne,Marc Moniez, Martin Rodriguez Monroy, Joseph Chevalier,}
\institute[IJCLab]{
  IJCLab,
  CNRS/IN2P3 \& Université Paris-Saclay,
  Orsay, France }
  
  % Caution  : This is not the affiliation to be used to sign articles. 
  % Please refer to the intranet : Bibliothèque/IST / Règles de publication
     
\date[IJCLab, Dec 16th 2022]{DESC PCWG, December 16th 2022}






% The main document with a few examples
%%%%%%%%%%%%%%%%%%%%%%%%%%%%%%%%%%%%%%%%%%%%%%%%%%%%%%%%%%%%%%%%%%%%%%%%%%%%%%
\begin{document}

%==================================================================================================================================
\begin{frame}
  \titlepage
\end{frame}
%==================================================================================================================================



%==================================================================================================================================
%\begin{frame}{Outline}
%  \tableofcontents
%\end{frame}
%==================================================================================================================================


% see example https://tex.stackexchange.com/questions/62902/text-column-in-array
% Define a text column environnement
%\usepackage{array,amsmath}
%\newcolumntype{L}{>$l<$}

%\begin{frame}
%\begin{equation*}
%    \begin{array}{L@{\quad}c@{{}={}}c}
%       some text:            & x^2 & x^2 \\
%       more thoughts:        & y^2 & y^2 \\
%       really deep thoughts: & z^2 & z^2
%    \end{array}
%\end{equation*}
%\end{frame}

%=================================================================================================================================
%\begin{frame}{Magnitude calibration of astronomical objects}
%\begin{alertblock}{Article Burke et al. 2017, Formula (16) }
%\begin{equation*}
%    \begin{array}{l@{{}{}}c c@{\quad}L}
%               m_b^{std} =   & -2.5 \log_{10}(ADU_b^{obs}) & \longrightarrow  & ADU photometric counts     \\
%                & +  2.5 \log_{10}\left(\mathbb{I}_0^{obs}(b)\right)  &\longrightarrow  &  {\color{red}0th order transmission term} \\
%                 & + ZPT^{AB}_{obs}  & \longrightarrow & Zero Point magnitude \\
%                 &  +  2.5 \log_{10} 
%	\left( 
%	\frac{\int_0^\infty F_\nu(\lambda) \times \phi_b^{obs}(\lambda) d\lambda }{\int_0^\infty F_\nu(\lambda) \times \phi_b^{std}(\lambda) d\lambda} 
%	\right) & \longrightarrow & {\color{blue}SED Color correction term}
%    \end{array}
%\end{equation*}
%\end{alertblock}
%{\footnotesize
%\begin{block}{Calibration quantities for each observation:}
%\begin{equation*}
%\begin{array}{L@{\quad}ccc}
%transmission integral : & \mathbb{I}_0^{obs}(b) & \equiv & \int_0^\infty S^{obs}_b(\lambda) \frac{d\lambda}{\lambda} \\
%zero point link to AB-flux unit/CCD : & ZPT^{AB}_{obs} & \equiv & 2.5 \log_{10} \left( \frac{A \Delta T F^{AB}}{gh}\right)
%\end{array} 
%\end{equation*}
%\end{block}
%\begin{itemize}
%\item Total observation transmission : $S_b^{obs}(\lambda) = S^{atm}(\lambda) \times S_b^{inst}(\lambda)$
%\item Normalized passband : $\phi_b^{obs} (\lambda,t) = \frac{S_b^{obs}(\lambda,t)\frac{1}{\lambda}}{\int_0^\infty S^{obs}_b(\lambda,t) \frac{d\lambda}{\lambda}}$
%\end{itemize}
%}
%\end{frame}
%=================================================================================================================================


%%%%%%%%%%%%%%%%%%%%%%%%%%%%%%%%%%%%%%%%%%%%%%%%%%%%%%%%%%%%%%%%%%%%%%%%%%%%%%%%%
\section{Motivation}
%https://arxiv.org/abs/1706.01542
\begin{frame}\sectionpage\end{frame}
%%%%%%%%%%%%%%%%%%%%%%%%%%%%%%%%%%%%%%%%%%%%%%%%%%%%%%%%%%%%%%%%%%%%%%%


%=================================================================================
\begin{frame}{Motivation}
\begin{itemize}
\item We are attending DESC group on Photometric Corrections,
\item We participate or wish to participate in comissioning of Rubin project in that PC field, 
\item We may be involved or not in measuring $S^{atm}(\lambda)$ and $S^{inst}(\lambda)$, or building the star monster catalog, or other stuff on Auxtel (DIMM, Weather,...),
\item We have science goal in DESC, PhotoZ or SN, or other,
\item We may need to make a more obvious link between our current actions and our science goals,
\item We may increase or prolongate our contribution further to prepare Rubin LSST project through our participation in commissioning,
\item We would-like to understand what we concreately could do, which kind of knowledge we need to get,
\end{itemize}
\end{frame}
%===============================================================================

%===============================================================================
\begin{frame}
\begin{itemize}
\item Propose a series of Onboarding talks open to any presenter (expert or not expert), willing to help in better understanding this PC subject, suggest actions.
\item Goal is to learn ...
\item Goal is to trigger interesting open discussions
\item Goal is to clear unobvious aspects
\item Today propose a first part on the review on article on FGCM : \href{https://arxiv.org/abs/1706.01542}{FORWARD GLOBAL PHOTOMETRIC CALIBRATION OF THE DARK ENERGY SURVEY, Burke et al. 2017}
\item More to come in 2023, when ready, when we want or when we can.
\end{itemize}
\end{frame}
%===============================================================================

%%%%%%%%%%%%%%%%%%%%%%%%%%%%%%%%%%%%%%%%%%%%%%%%%%%%%%%%%%%%%%%%%%%%%%%%%%%%%%%%%%%%%%%%%
\section{Photometric correction for standard magnitude}
\begin{frame}\sectionpage\end{frame}
%%%%%%%%%%%%%%%%%%%%%%%%%%%%%%%%%%%%%%%%%%%%%%%%%%%%%%%%%%%%%%%%%%%%%%%%%%%%%%%%%%%%%%%%%


%=================================================================================================================================
\begin{frame}{Standard magnitude calibration of objects}
\begin{alertblock}{Article Burke et al. 2017, Formula (16)}
\begin{equation*}
    \begin{array}{l@{{}{}}c c@{\quad}L}
               m_b^{std} =   & -2.5 \log_{10}(ADU_b^{obs}) & \longrightarrow  & ADU photometric counts     \\
                & +  2.5 \log_{10}\left(\frac{\mathbb{I}_0^{obs}(b)}{\mathbb{I}_0^{std}(b)}\right)  &\longrightarrow  &  {\color{red}0th order transmission term} \\
                 & + ZPT^{AB}_{obs} + 2.5 \log_{10}\left(\mathbb{I}_0^{std}(b)\right) & \longrightarrow & Zero Point magnitude \\
                 &  +  2.5 \log_{10} 
	\left( 
	\frac{\int_0^\infty F_\nu(\lambda) \times \phi_b^{obs}(\lambda) d\lambda }{\int_0^\infty F_\nu(\lambda) \times \phi_b^{std}(\lambda) d\lambda} 
	\right) & \longrightarrow & {\color{blue}SED Color correction term}
    \end{array}
\end{equation*}
\end{alertblock}
{\footnotesize
\begin{block}{Calibration quantities for each observation:}
\begin{equation*}
\begin{array}{L@{\quad}ccc}
transmission integral : & \mathbb{I}_0^{obs}(b) & \equiv & \int_0^\infty S^{obs}_b(\lambda) \frac{d\lambda}{\lambda} \\
zero point link to AB-flux unit/CCD : & ZPT^{AB}_{obs} & \equiv & 2.5 \log_{10} \left( \frac{A \Delta T F^{AB}}{gh}\right)
\end{array} 
\end{equation*}
\end{block}
\begin{itemize}
\item Total transmission : $S_b^{obs}(\lambda) = S^{atm-obs}(\lambda) \times S_b^{inst-obs}(\lambda)$
\item Normalized passband : $\phi_b^{c=obs,std} (\lambda,t) = \frac{S_b^{c}(\lambda,t)\frac{1}{\lambda}}{\int_0^\infty S^{c}_b(\lambda,t) \frac{d\lambda}{\lambda}}$
\end{itemize}
}
\end{frame}
%========================================================================================================================




% Slide 2
%=================================================================================================================================
\begin{frame}{Interpretation of correction terms to $m_{std}$}
{\footnotesize
The standard transmission is chosen apriory thus $\mathbb{I}_0^{std}(b)$ is calculated
}
{\footnotesize
\begin{tabular}{llll} \hline
a)&0th order corr term : & relative attenuation in $b$ & $2.5 \log_{10}\left(\frac{\mathbb{I}_0^{obs}(b)}{\mathbb{I}_0^{std}(b)}\right)$ \\
b)&1st order corr term : & SED color correction& $ 2.5 \log_{10} 
	\left( 
	\frac{\int_0^\infty F_\nu(\lambda) \times \phi_b^{obs}(\lambda) d\lambda }{\int_0^\infty F_\nu(\lambda) \times \phi_b^{std}(\lambda) d\lambda} 
	\right)$ \\
c)&constant term/CCD/exp: & Zero point to measure & $ZPT^{AB}_{obs}\equiv  2.5 \log_{10} \left( \frac{A \Delta T F^{AB}}{g_{el}h}\right)$ \\
d)&constant term/band : & Calculated & $2.5 \log_{10}\left(\mathbb{I}_0^{std}(b)\right)$ \\  \hline
\end{tabular}


\begin{itemize}
\item a) requires atmospheric transparency either by direct measurement with Auxtel or indirectly by fitting air-transparency in LSST-FGCM atmospheric model or combining both,
\item b) requires atmospheric transparency as well:
\begin{itemize}
\item stars : requires the magnitude color type of the stars 
\item galaxies and other : requires to classify by colors (by example for PhotoZ
\end{itemize}
\item c) Set the flux scale unit per CCD per exposure and is band independent.
\begin{itemize}
\item absorbs grey component common to all bands, (ex clouds, ageing,...)
\item measured with the contraints provided by the calibration star in the Monster Catalog
\end{itemize}
\end{itemize}
}

\end{frame}
%========================================================================================================================


%%%%%%%%%%%%%%%%%%%%%%%%%%%%%%%%%%%%%%%%%%%%%%%%%%%%%%%%%%%%%%%%%%%%%%%%%%%%%%%%%%%%%%%%%
\section{Order 0 band correction term}
\begin{frame}\sectionpage\end{frame}
%%%%%%%%%%%%%%%%%%%%%%%%%%%%%%%%%%%%%%%%%%%%%%%%%%%%%%%%%%%%%%%%%%%%%%%%%%%%%%%%%%%%%%%%%



%========================================================================================================================
\begin{frame}{Relative ratio of integrals}
Focusing on atmosphere component variations defined as
$\theta^{atm}= PWV, O_3, aerosols$
\begin{alertblock}{}
\begin{equation*}
\left(\frac{\mathbb{I}_0^{obs}(b,z_{obs},\theta_{obs}^{atm})}{\mathbb{I}_0^{std}(b,z_{std},\theta_{std}^{atm})}\right) =
\underbrace{\left(\frac{\mathbb{I}_0^{obs}(b,z_{obs},\theta_{obs}^{atm})}{\mathbb{I}_0^{std}(b,z_{obs},\theta_{std}^{atm})}\right)}_{1:ratio\;at \; same \; z_{obs}\; \& \; \neq \; \theta^{atm}} \times \underbrace{\left(\frac{\mathbb{I}_0^{std}(b,z_{obs},\theta_{std}^{atm})}{\mathbb{I}_0^{std}(b,z_{std},\theta_{std}^{atm})}\right)}_{2:calculable \; ratio \;at \; z_{obs}\neq z_{std} }
\end{equation*}
\end{alertblock}
\begin{itemize}
\item 1 : $\left(\frac{\mathbb{I}_0^{obs}(b,z_{obs},\theta_{obs}^{atm})}{\mathbb{I}_0^{std}(b,z_{obs},\theta_{std}^{atm})}\right)$ : numerator obtained by the {\bf measurement of atmospheric transmission for each exposure}, denominator calculated at observation $z_{obs}$ (small corrections at ~10 mmag level),
\item 2 : $\left(\frac{\mathbb{I}_0^{std}(b,z_{obs},\theta_{std}^{atm})}{\mathbb{I}_0^{std}(b,z_{std},\theta_{std}^{atm})}\right)$ : calculable ratio assuming knowledge of atmospheric model (large corrections at ~100 mmag level). 
\end{itemize}


\end{frame}
%========================================================================================================================




%========================================================================================================================
%\begin{frame}{Order 0 magnitude corrections at different airmass}
%\begin{itemize}
%\item standard transmission, standard atmosphere at airmass $z=z_{std}$
%\item observed transmission, standard atmosphere at airmass $z_{obs} \ne z_{std}$
%\end{itemize}

%\begin{columns}
%\begin{column}{0.6\textwidth}
%\includegraphics[width=8cm,angle=0]{figs/PCCorr/fig1_PCCorrOrder0th_airmass.png}
%\end{column}
%\begin{column}{0.4\textwidth}  %%<--- here
%\centering
%\includegraphics[width=4cm,angle=0]{figs/PCCorr/fig2_PCCorrOrder0th_airmass.png}
%\end{column}
%\end{columns}  
%\end{frame}
%========================================================================================================================

%========================================================================================================================
\begin{frame}{Order 0 corrections at different airmass}

\begin{tabular}{cc}
\includegraphics[width=6cm,height=6.5cm,angle=0]{figs/PCCorr/fig1_PCCorrOrder0th_airmass.png} &
\includegraphics[width=5.5cm,height=4cm,angle=0]{figs/PCCorr/fig2_PCCorrOrder0th_airmass.png}
\end{tabular}
\end{frame}
%========================================================================================================================

%========================================================================================================================
\begin{frame}{Order 0 corrections at different PWV}

\begin{tabular}{cc}
\includegraphics[width=6cm,height=6.5cm,angle=0]{figs/PCCorr/fig1_PCCorrOrder0th_PWV.png} &
\includegraphics[width=5.5cm,height=6cm,angle=0]{figs/PCCorr/fig2_PCCorrOrder0th_PWV.png}
\end{tabular}
\end{frame}
%========================================================================================================================


%========================================================================================================================
\begin{frame}{Order 0 corrections at different VAOD}

\begin{tabular}{cc}
\includegraphics[width=6cm,height=6.5cm,angle=0]{figs/PCCorr/fig1_PCCorrOrder0th_aer.png} &
\includegraphics[width=5.5cm,height=6cm,angle=0]{figs/PCCorr/fig2_PCCorrOrder0th_aer.png}
\end{tabular}
\end{frame}
%========================================================================================================================




%%%%%%%%%%%%%%%%%%%%%%%%%%%%%%%%%%%%%%%%%%%%%%%%%%%%%%%%%%%%%%%%%%%%%%%%%%%%%%%%%%%%%%%%%
\section{Order 1 Color correction term}
\begin{frame}\sectionpage\end{frame}
%%%%%%%%%%%%%%%%%%%%%%%%%%%%%%%%%%%%%%%%%%%%%%%%%%%%%%%%%%%%%%%%%%%%%%%%%%%%%%%%%%%%%%%%%

%=====================================================================================
\begin{frame}{Effective normalized passband}
\begin{tabular}{cc}
\includegraphics[width=5.5cm,height=3.5cm,angle=0]{figs/PCCorr/fig1_PCCorrOrder1_airmass.png}
&
\includegraphics[width=5.5cm,height=3.5cm,angle=0]{figs/PCCorr/fig2_PCCorrOrder1_PWV.png} \\
 $\boxed{\phi_b^{c=obs,std} (\lambda,t) = \frac{S_b^{c}(\lambda,t)\frac{1}{\lambda}}{\int_0^\infty S^{c}_b(\lambda,t) \frac{d\lambda}{\lambda}}}$ & \includegraphics[width=5.5cm,height=3.5cm,angle=0]{figs/PCCorr/fig3_PCCorrOrder1_aer.png}
\end{tabular}
\end{frame}
%=======================================================================================

%==========================================================================================================================
\begin{frame}{Transmission moments definitions}
\begin{block}{Transmission moments definitions}

\begin{eqnarray}
\mathbb{I}_0^i(b) & = & \int_0^{\infty} S_b^i(\lambda) \frac{d\lambda}{\lambda} \\
\mathbb{I}_1^i(b) & = & \int_0^{\infty} (\lambda-\lambda_b) S_b^i(\lambda) \frac{d\lambda}{\lambda} \\
 \mathbb{I}_2^i(b) & = &\int_0^{\infty} (\lambda-\lambda_b)^2 S_b^i(\lambda) \frac{d\lambda}{\lambda} \\
\mathbb{I}^i_{10}(b)  & = &\frac{\mathbb{I}^i_1(b)}{\mathbb{I}^i_0(b)} \\
\mathbb{I}^i_{20}(b) & = & \frac{\mathbb{I}^i_2(b)}{\mathbb{I}^i_0(b)} 
\end{eqnarray}

with $i=obs$ or $i=std$
\end{block}
\end{frame}
%==========================================================================================================================






\subsection{Approximation for SED shape}
%==========================================================================================================================
\begin{frame}{Approximation for SED shape} 
\begin{exampleblock}{SED approximation as Taylor expansion (see FGCM DES article)}
\begin{eqnarray}
F_\nu(\lambda) & = & F_\nu(\lambda_b) \left(1 + f^\prime(\lambda_b)(\lambda-\lambda_b) + \frac{f^{\prime\prime}(\lambda_b)}{2}(\lambda-\lambda_b)^2 + \cdots \right) \\
f_\nu^\prime(\lambda) & \equiv & \frac{1}{F_\nu(\lambda)}\frac{dF_\nu(\lambda)}{d\lambda} \qquad \qquad \qquad
f_\nu^{\prime\prime}(\lambda)  \equiv  \frac{1}{F_\nu(\lambda)}\frac{d^2F_\nu(\lambda)}{d\lambda^2} \nonumber \\
\lambda_b & \equiv & \frac{\int_0^{\infty} \lambda \times S_b^{inst}(\lambda) \frac{d\lambda}{\lambda}}
{\int_0^{\infty} S_b^{inst}(\lambda) \frac{d\lambda}{\lambda}}
\end{eqnarray}

\begin{itemize}
\item $S_b^{obs}(\lambda) = S_b^{inst}(\lambda,t,x,y) \times S^{atm}(\lambda,t,alt,az)$
\end{itemize}
\end{exampleblock}
\end{frame}
%==========================================================================================================================


\subsection{Color correction term}
%==========================================================================================================================
\begin{frame}{Color correction term}
\begin{alertblock}{Color correction term}
\begin{eqnarray}
\Delta m_b (color) & \equiv  & 2.5 \log_{10} 
	\left( 
	\frac{\int_0^\infty F_\nu(\lambda) \times \phi_b^{obs}(\lambda) d\lambda }{\int_0^\infty F_\nu(\lambda) \times \phi_b^{std}(\lambda) d\lambda} 
	\right)  \nonumber \\
          & \equiv  &  2.5 \log_{10}\left( 
\frac{1 + f_\nu^\prime(\lambda_b)\mathbb{I}^{obs}_{10}(b)  +\frac{f_\nu^{\prime\prime}(\lambda_b)}{2}\mathbb{I}_{20}^{obs}(b)}
{1 + f_\nu^\prime(\lambda_b)\mathbb{I}^{std}_{10}(b)  +\frac{f_\nu^{\prime\prime}(\lambda_b)}{2}\mathbb{I}_{20}^{std}(b)}\right) \nonumber \\
& \simeq &  + 1.087\left( f_\nu^\prime(\lambda_b) \Delta \mathbb{I}_{10}(b) +
          \frac{f_\nu^{\prime\prime}(\lambda_b)}{2}\Delta \mathbb{I}_{20}(b)  \right) \nonumber
%          & & \left. - \left. \frac{1}{2}\left( f_\nu^\prime(\lambda_b) \Delta \mathbb{I}_{10}(b) \right)^2  \right) \nonumber
\end{eqnarray}
\end{alertblock}

\begin{block}{Moments difference definition}
\begin{eqnarray}
\Delta \mathbb{I}_{10}(b) & = &  \mathbb{I}_{10}^{obs}(b)  -  \mathbb{I}_{10}^{std}(b) \\
\Delta \mathbb{I}_{20}(b) & = &   \mathbb{I}_{20}^{obs}(b)  -  \mathbb{I}_{20}^{std}(b) 
\end{eqnarray}
\end{block}
\end{frame}
%==========================================================================================================================



%==========================================================================
\begin{frame}{Moments difference definition $\Delta \mathbb{I}_{10}(b)$}
\begin{tabular}{cc}
\includegraphics[width=5.5cm,height=3.5cm,angle=0]{figs/PCCorr/fig1_II10diff_airmass.png}
&
\includegraphics[width=5.5cm,height=3.5cm,angle=0]{figs/PCCorr/fig2_II10diff_PWV.png} \\

 $\boxed{\Delta \mathbb{I}_{10}(b)  =   \mathbb{I}_{10}^{obs}(b)  -  \mathbb{I}_{10}^{std}(b)}$ & \includegraphics[width=5.5cm,height=3.5cm,angle=0]{figs/PCCorr/fig3_II10diff_aer.png}
\end{tabular}
\end{frame}
%==========================================================================


%==========================================================================
\begin{frame}{Moments difference definition $\Delta \mathbb{I}_{20}(b)$}
\begin{tabular}{cc}
\includegraphics[width=5.5cm,height=3.5cm,angle=0]{figs/PCCorr/fig1_II20diff_airmass.png}
&
\includegraphics[width=5.5cm,height=3.5cm,angle=0]{figs/PCCorr/fig2_II20diff_PWV.png} \\

 $\boxed{\Delta \mathbb{I}_{20}(b)  =   \mathbb{I}_{20}^{obs}(b)  -  \mathbb{I}_{20}^{std}(b)}$ & \includegraphics[width=5.5cm,height=3.5cm,angle=0]{figs/PCCorr/fig3_II20diff_aer.png}
\end{tabular}
\end{frame}
%==========================================================================








%%%%%%%%%%%%%%%%%%%%%%%%%%%%%%%%%%%%%%%%%%%%%%%%%%%%%%%%%%%%%%%%%%%%%%%%%%%%%%%%%%%%%%%%%%
\section{Next topics for next year}
\begin{frame}\sectionpage\end{frame}
%%%%%%%%%%%%%%%%%%%%%%%%%%%%%%%%%%%%%%%%%%%%%%%%%%%%%%%%%%%%%%%%%%%%%%%%%%%%%%%%%%%%%%%%%



\begin{frame}{Next subjects to come in 2023}
\begin{itemize}
\item Photometric Color corrections on stars or objects of know SED (not redshifted)
\item Photometric Color corrections on redshifted objects
\item For what purpose the Monster catalog is needed ?
\item Measurement of zero points at Rubin-LSST ?
\item Instrument corrections $S^{inst}(\lambda,t)$ ?
\item Photometric Aperture errors \& Rubin Cadence and Observing Conditions ?
\item DM calibration pipeline : what are the data-products required, what are the task/algorithm needed, and where in the whole pipeline. 
\item What can be done \& checked during (Auxtel) commissioning ?
\item What else ?
\end{itemize}
\end{frame}


\begin{frame}{Rubin-LSST calibration plan}
\includegraphics[width=0.95\paperwidth]{figs/calib/LSSTCalibrationPlan_v2.png}
\end{frame}




%==========================================================================================================================
\begin{frame}
\begin{center}
{\usebeamerfont{frametitle}

\LARGE \alert{End of part 1}}

\end{center}

\end{frame}
%==========================================================================================================================










 
\end{document}


