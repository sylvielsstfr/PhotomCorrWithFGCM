% Copyright 2022 by Sébastien Descotes-Genon
%
% This file may be distributed and/or modified
%
% 1. under the LaTeX Project Public License and/or
% 2. under the GNU Public License.
%


\documentclass{beamer}

\usepackage{array,amsmath}
\newcolumntype{L}{>$l<$}



% Setup appearance:

\usetheme{IJCLab}

\pgfdeclareimage[height=2cm]{LogoIJCLab}{example-fig/LogoIJCLab}
\pgfdeclareimage[height=1.3cm]{LogoCNRS}{example-fig/LogoCNRS}
\pgfdeclareimage[height=1cm]{LogoUPSay}{example-fig/LogoUPSay}


\usepackage[english]{babel}
\usepackage[T1]{fontenc}


% Author, Title, etc.

\title[Atmospheric transparency at LSST Obs]{\large Photometric Calibration at Rubin-LSST observatory \\
Estimation of atmospheric transparency \\ Correction for Water \& Aerosol variations\\
{\small Univers du Pôle A2C}}

\titlegraphic{\begin{center}\qquad\pgfuseimage{LogoCNRS}\qquad\quad\qquad\quad\pgfuseimage{LogoIJCLab}\qquad\qquad\pgfuseimage{LogoUPSay}\end{center}}

\author[S. Dagoret-Campagne]{
Sylvie Dagoret-Campagne,Marc Moniez, Martin Rodriguez Monroy, Joseph Chevalier,
Jeremy Neveu
}
\institute[IJCLab]{
  IJCLab,
  CNRS/IN2P3 \& Université Paris-Saclay,
  Orsay, France }
  
  % Caution  : This is not the affiliation to be used to sign articles. 
  % Please refer to the intranet : Bibliothèque/IST / Règles de publication
     
\date[IJCLab, Dec 14th 2022]{IJCLab, pôle A2C, December 14th 2022}


% The main document with a few examples

\begin{document}

%==================================================================================================================================
\begin{frame}
  \titlepage
\end{frame}
%==================================================================================================================================



%==================================================================================================================================
\begin{frame}{Motivations}
\begin{itemize}
\item Sources/Object standard magnitude estimation from instrumental magnitude $\rightarrow$ allowing coaddition of sigle epoch measurements
\item use of time dependent epoch transmissions (instrument + atmosphere)
\item photometric correction depending on poorly known objects SED
\item to be implemented inside DM Rubin-LSST science pipelines 
\item participation of DESC technical groups (SAWG,PCWG,PSF), namely during commissioning phase,
\item need participation of Science groups (PhotoZ, and objects groups like SN, and other astrophysical groups),
\end{itemize}
\end{frame}
%==================================================================================================================================

%==================================================================================================================================
\begin{frame}{Outline}
  \tableofcontents
\end{frame}
%==================================================================================================================================


% see example https://tex.stackexchange.com/questions/62902/text-column-in-array
% Define a text column environnement
%\usepackage{array,amsmath}
%\newcolumntype{L}{>$l<$}

%\begin{frame}
%\begin{equation*}
%    \begin{array}{L@{\quad}c@{{}={}}c}
%       some text:            & x^2 & x^2 \\
%       more thoughts:        & y^2 & y^2 \\
%       really deep thoughts: & z^2 & z^2
%    \end{array}
%\end{equation*}
%\end{frame}

%=================================================================================================================================
\begin{frame}{Magnitude calibration of astronomical objects}<presentation:0>
\begin{alertblock}{Single visit object magnitude to be coadded in band $b$}
\begin{equation*}
    \begin{array}{l@{{}{}}c c@{\quad}L}
               m_b^{std} =   & -2.5 \log_{10}(ADU_b^{obs}) & \longrightarrow  & ADU photometric counts     \\
                & +  2.5 \log_{10}\left(\mathbb{I}_0^{obs}(b)\right)  &\longrightarrow  &  {\color{red}0th order transmission term} \\
                 & + ZPT^{AB}_{obs}  & \longrightarrow & Zero Point magnitude \\
                 &  +  2.5 \log_{10} 
	\left( 
	\frac{\int_0^\infty F_\nu(\lambda) \times \phi_b^{obs}(\lambda) d\lambda }{\int_0^\infty F_\nu(\lambda) \times \phi_b^{std}(\lambda) d\lambda} 
	\right) & \longrightarrow & {\color{blue}SED Color correction term}
    \end{array}
\end{equation*}
\end{alertblock}
{\footnotesize
\begin{block}{Calibration quantities for each observation:}
\begin{equation*}
\begin{array}{L@{\quad}ccc}
transmission integral : & \mathbb{I}_0^{obs}(b) & \equiv & \int_0^\infty S^{obs}_b(\lambda) \frac{d\lambda}{\lambda} \\
zero point link to AB-flux unit/CCD : & ZPT^{AB}_{obs} & \equiv & 2.5 \log_{10} \left( \frac{A \Delta T F^{AB}}{gh}\right)
\end{array} 
\end{equation*}
\end{block}
\begin{itemize}
\item Total observation transmission : $S_b^{obs}(\lambda) = S^{atm}(\lambda) \times S_b^{inst}(\lambda)$
\item Normalized passband : $\phi_b^{obs} (\lambda,t) = \frac{S_b^{obs}(\lambda,t)\frac{1}{\lambda}}{\int_0^\infty S^{obs}_b(\lambda,t) \frac{d\lambda}{\lambda}}$
\end{itemize}
}
\end{frame}
%=================================================================================================================================


%=================================================================================================================================
\begin{frame}{Magnitude calibration of astronomical objects}
\begin{alertblock}{Single visit object magnitude to be coadded in band $b$}
\begin{equation*}
    \begin{array}{l@{{}{}}c c@{\quad}L}
               m_b^{std} =   & -2.5 \log_{10}(ADU_b^{obs}) & \longrightarrow  & ADU photometric counts     \\
                & +  2.5 \log_{10}\left(\frac{\mathbb{I}_0^{obs}(b)}{\mathbb{I}_0^{std}(b)}\right)  &\longrightarrow  &  {\color{red}0th order transmission term} \\
                 & + ZPT^{AB}_{obs} + 2.5 \log_{10}\left(\mathbb{I}_0^{std}(b)\right) & \longrightarrow & Zero Point magnitude \\
                 &  +  2.5 \log_{10} 
	\left( 
	\frac{\int_0^\infty F_\nu(\lambda) \times \phi_b^{obs}(\lambda) d\lambda }{\int_0^\infty F_\nu(\lambda) \times \phi_b^{std}(\lambda) d\lambda} 
	\right) & \longrightarrow & {\color{blue}SED Color correction term}
    \end{array}
\end{equation*}
\end{alertblock}
{\footnotesize
\begin{block}{Calibration quantities for each observation:}
\begin{equation*}
\begin{array}{L@{\quad}ccc}
transmission integral : & \mathbb{I}_0^{obs}(b) & \equiv & \int_0^\infty S^{obs}_b(\lambda) \frac{d\lambda}{\lambda} \\
zero point link to AB-flux unit/CCD : & ZPT^{AB}_{obs} & \equiv & 2.5 \log_{10} \left( \frac{A \Delta T F^{AB}}{gh}\right)
\end{array} 
\end{equation*}
\end{block}
\begin{itemize}
\item Total transmission : $S_b^{obs}(\lambda) = S^{atm-obs}(\lambda) \times S_b^{inst-obs}(\lambda)$
\item Normalized passband : $\phi_b^{c=obs,std} (\lambda,t) = \frac{S_b^{c}(\lambda,t)\frac{1}{\lambda}}{\int_0^\infty S^{c}_b(\lambda,t) \frac{d\lambda}{\lambda}}$
\end{itemize}
}
\end{frame}
%========================================================================================================================

%========================================================================================================================
\begin{frame}{Magnitude corrections for non standard obs}
\begin{itemize}
\item standard transmission, standard atmosphere at airmass $z=z_{std}$
\item observed transmission, standard atmosphere at airmass $z_{obs} \ne z_{std}$
\end{itemize}

\begin{columns}
\begin{column}{0.3\textwidth}
SED-color independent correction :\\ only air transmission and telescope throughput
\end{column}
\begin{column}{0.65\textwidth}  %%<--- here
%\centering
    \includegraphics[width=8cm,angle=0]{figs/PCCorr/PCCorrOrder0th_airmass.png}
\end{column}
\end{columns}
   
\end{frame}
%========================================================================================================================



\begin{frame}{Relative ratio of integrals}
Focusing on atmosphere component variations defined as
$\theta^{atm}= PWV, O_3, aerosols$
\begin{alertblock}{}
\begin{equation*}
\left(\frac{\mathbb{I}_0^{obs}(b,z_{obs},\theta_{obs}^{atm})}{\mathbb{I}_0^{std}(b,z_{std},\theta_{std}^{atm})}\right) =
\underbrace{\left(\frac{\mathbb{I}_0^{obs}(b,z_{obs},\theta_{obs}^{atm})}{\mathbb{I}_0^{std}(b,z_{obs},\theta_{std}^{atm})}\right)}_{1:ratio\;at \; same \; z_{obs}\; \& \; \neq \; \theta^{atm}} \times \underbrace{\left(\frac{\mathbb{I}_0^{std}(b,z_{obs},\theta_{std}^{atm})}{\mathbb{I}_0^{std}(b,z_{std},\theta_{std}^{atm})}\right)}_{2:calculable \; ratio \;at \; z_{obs}\neq z_{std} }
\end{equation*}
\end{alertblock}
\begin{itemize}
\item 1 : $\left(\frac{\mathbb{I}_0^{obs}(b,z_{obs},\theta_{obs}^{atm})}{\mathbb{I}_0^{std}(b,z_{obs},\theta_{std}^{atm})}\right)$ : numerator obtained by the {\bf measurement of atmospheric transmission for each exposure}, denominator calculated at observation $z_{obs}$ (small corrections at ~10 mmag level),
\item 2 : $\left(\frac{\mathbb{I}_0^{std}(b,z_{obs},\theta_{std}^{atm})}{\mathbb{I}_0^{std}(b,z_{std},\theta_{std}^{atm})}\right)$ : calculable ratio assuming knowledge of atmospheric model (large corrections at ~100 mmag level). 
\end{itemize}


\end{frame}




\subsection{Approximation for SED shape}
%==========================================================================================================================
\begin{frame}{Approximation for SED shape} 
\begin{exampleblock}{SED approximation as Taylor expansion}
\begin{eqnarray}
F_\nu(\lambda) & = & F_\nu(\lambda_b) \left(1 + f^\prime(\lambda_b)(\lambda-\lambda_b) + \frac{f^{\prime\prime}(\lambda_b)}{2}(\lambda-\lambda_b)^2 + \cdots \right) \\
f_\nu^\prime(\lambda) & \equiv & \frac{1}{F_\nu(\lambda)}\frac{dF_\nu(\lambda)}{d\lambda} \qquad \qquad \qquad
f_\nu^{\prime\prime}(\lambda)  \equiv  \frac{1}{F_\nu(\lambda)}\frac{d^2F_\nu(\lambda)}{d\lambda^2} \nonumber \\
\lambda_b & \equiv & \frac{\int_0^{\infty} \lambda \times S_b^{inst}(\lambda) \frac{d\lambda}{\lambda}}
{\int_0^{\infty} S_b^{inst}(\lambda) \frac{d\lambda}{\lambda}}
\end{eqnarray}
\begin{itemize}
\item $S_b^{obs}(\lambda) = S_b^{inst}(\lambda,t,x,y) \times S^{atm}(\lambda,t,alt,az)$
\end{itemize}
\end{exampleblock}
\end{frame}
%==========================================================================================================================

\subsection{Summary of definitions on interpretable standard magnitude}
%==========================================================================================================================
\begin{frame}{Summary of definitions}
\begin{alertblock}{Decomposition of standard magnitude}
\begin{eqnarray}
m_b^{std} & = & -2.5 \log_{10}(C_b)  \nonumber \\
          &   &  + 2.5 \log_{10}(\mathbb{I}_0^{obs}(b)) + ZPT^{AB}  \nonumber \\
          &   &  + 2.5 \log_{10}\left( 
\frac{1 + f_\nu^\prime(\lambda_b)\mathbb{I}^{obs}_{10}(b)  +\frac{f_\nu^{\prime\prime}(\lambda_b)}{2}\mathbb{I}_{20}^{obs}(b)}
{1 + f_\nu^\prime(\lambda_b)\mathbb{I}^{std}_{10}(b)  +\frac{f_\nu^{\prime\prime}(\lambda_b)}{2}\mathbb{I}_{20}^{std}(b)}\right)
\end{eqnarray}
\end{alertblock}
\end{frame}
%==========================================================================================================================


%==========================================================================================================================
\begin{frame}{Transmission moments definitions}
\begin{block}{Transmission moments definitions}

\begin{eqnarray}
\mathbb{I}_0^i(b) & = & \int_0^{\infty} S_b^i(\lambda) \frac{d\lambda}{\lambda} \\
\mathbb{I}_1^i(b) & = & \int_0^{\infty} (\lambda-\lambda_b) S_b^i(\lambda) \frac{d\lambda}{\lambda} \\
 \mathbb{I}_2^i(b) & = &\int_0^{\infty} (\lambda-\lambda_b)^2 S_b^i(\lambda) \frac{d\lambda}{\lambda} \\
\mathbb{I}^i_{10}(b)  & = &\frac{\mathbb{I}^i_1(b)}{\mathbb{I}^i_0(b)} \\
\mathbb{I}^i_{20}(b) & = & \frac{\mathbb{I}^i_2(b)}{\mathbb{I}^i_0(b)} 
\end{eqnarray}

with $i=obs$ or $i=std$
\end{block}
\end{frame}
%==========================================================================================================================




\subsection{SED shape correction}
%==========================================================================================================================
\begin{frame}{Approximation for SED shape correction}
\begin{alertblock}{SED shape ($f^\prime_\nu,f^{\prime\prime}_\nu$) color correction}
\begin{eqnarray}
m_b^{std} & = & -2.5 \log_{10}(C_b)  \nonumber \\
          &   &  + 2.5 \log_{10}(\mathbb{I}_0^{obs}(b)) + ZPT^{AB}  \nonumber \\
          &   &  + 1.087\left( f_\nu^\prime(\lambda_b) \Delta \mathbb{I}_{10}(b) +
          \frac{f_\nu^{\prime\prime}(\lambda_b)}{2}\Delta \mathbb{I}_{20}(b) \right. \nonumber \\
         & & - \left. \frac{1}{2}\left( f_\nu^\prime(\lambda_b) \Delta \mathbb{I}_{10}(b) \right)^2             
          \right)
\end{eqnarray}
\end{alertblock}
\begin{block}{Moments difference definition}
\begin{eqnarray}
\Delta \mathbb{I}_{10}(b) & = &  \mathbb{I}_{10}^{obs}(b)  -  \mathbb{I}_{10}^{std}(b) \\
\Delta \mathbb{I}_{20}(b) & = &   \mathbb{I}_{20}^{obs}(b)  -  \mathbb{I}_{20}^{std}(b) 
\end{eqnarray}
\end{block}
\end{frame}
%==========================================================================================================================



\subsection{Bias on Magnitude for SED shape correction}
%==========================================================================================================================
\begin{frame}{Bias on Magnitude for SED shape correction}
\begin{alertblock}{Error on Magnitude for SED-shape correction}
\begin{eqnarray}
\Delta m & = & \left| 2.5 \log_{10}\left(
\frac{\mathbb{I}_0^{std}(b)}
{\mathbb{I}_0^{obs}(b)}\right) +
2.5 \log_{10} 
	\left( 
	\frac{\int_0^\infty F_\nu(\lambda) \times S_b^{obs}(\lambda) \frac{d\lambda}{\lambda} }{\int_0^\infty F_\nu(\lambda) \times S_b^{std}(\lambda) \frac{d\lambda}{\lambda}} 
	\right) \right. \nonumber \\
    & & \left. - 1.087\left( f_\nu^\prime(\lambda_b) \Delta \mathbb{I}_{10}(b) +
          \frac{f_\nu^{\prime\prime}(\lambda_b)}{2}\Delta \mathbb{I}_{20}(b) \right. 
         - \left. \frac{1}{2}\left( f_\nu^\prime(\lambda_b) \Delta \mathbb{I}_{10}(b) \right)^2             
          \right)\right| \nonumber \\
\Delta m & = & \left| 
2.5 \log_{10} 
	\left( 
	\frac{\int_0^\infty F_\nu(\lambda) \times \phi_b^{obs}(\lambda) d\lambda}{\int_0^\infty F_\nu(\lambda) \times \phi_b^{std}(\lambda)d\lambda} 
	\right) \right. \nonumber \\
    & & \left. - 1.087\left( f_\nu^\prime(\lambda_b) \Delta \mathbb{I}_{10}(b) +
          \frac{f_\nu^{\prime\prime}(\lambda_b)}{2}\Delta \mathbb{I}_{20}(b) \right. 
         - \left. \frac{1}{2}\left( f_\nu^\prime(\lambda_b) \Delta \mathbb{I}_{10}(b) \right)^2             
          \right)\right| \nonumber
\end{eqnarray}  
\end{alertblock}
\end{frame}
%==========================================================================================================================

%==========================================================================================================================
\begin{frame}{Bias on Magnitude for SED shape correction}
\begin{block}{Error on Magnitude for SED-shape correction}
\begin{eqnarray}
\Delta \mathbb{I}_{i0}(b) & = &  \mathbb{I}_{i0}^{obs}(b,z_{obs},aer_{obs},pwv_{obs}) - \mathbb{I}_{i0}^{std}(b,z_{std},aer_{std},pwv_{std}) \nonumber \\
& = & \left(\mathbb{I}_{i0}^{obs}(b,z_{obs},aer_{obs},pwv_{obs}) - \mathbb{I}_{i0}^{std}(b,z_{obs},aer_{std},pwv_{std})\right) +  \nonumber \\
& & \left(\mathbb{I}_{i0}^{std}(b,z_{obs},aer_{std},pwv_{std}) - \mathbb{I}_{i0}^{std}(b,z_{std},aer_{std},pwv_{std})\right)  \nonumber 
\end{eqnarray}
\end{block}

\begin{alertblock}{Linearity of corrections - standard atmosphere at different airmass}
\begin{eqnarray}
\Delta \mathbb{I}_{i0}(b) & = & \left(\mathbb{I}_{i0}^{obs}(b,z_{obs}) - \mathbb{I}_{i0}^{std}(b,z_{obs})\right) + \frac{\partial}{\partial z}\mathbb{I}_{i0}^{std}(b,z_{std})(z_{obs}-z_{std}) + \cdots \nonumber
\end{eqnarray}
\end{alertblock}
\begin{itemize}
\item $\mathbb{I}_{i0}^{obs}(b,z_{obs})$ : measured
\item $\mathbb{I}_{i0}^{std}(b,z_{obs})$ and $\frac{\partial}{\partial z}\mathbb{I}_{i0}^{std}(b,z_{std})$: from standard atmospheric model
\end{itemize}
\end{frame}
%==========================================================================================================================


\subsection{What is FGCM  in DES\&LSST ?}
%================================================================================================================
\begin{frame}{Forward Global Calibration Model {\small (DES\& LSST)}}
\begin{itemize}
\item From a reference catalog of calibration stars $j$ wih known $\overline{m_b^{std}(j)}$
\item Optimize the following $\chi^2$ in band $b$ ($i$ exposure, $j$ star-object, $\sigma_{phot}(i,j)$, photometric error):
\begin{equation} 
\chi^2_b = 
\sum_{(i,j)} \frac{ \left(m_b^{std}(i,j) - \overline{m_b^{std}(j)} \right)^2}{\sigma_{phot}^2(i,j)}
\end{equation}
\item With the measured magnitude in LSST is~:
\begin{eqnarray}
m_b^{std}(i,j) & = & -2.5\log_{10}(C_b^{i,j}) +2.5\log_{10}(\mathbb{I}_0^{obs,i}(b)) +ZPT^{AB}(i)  \nonumber \\
& + & 2.5 \log_{10}\left( \frac{1+f_\nu^{\prime}(\lambda_b)(b)\mathbb{I}_{10}^{obs,i}(b)}{1+f_\nu^{\prime}(\lambda_b)(b)\mathbb{I}_{10}^{std}(b)}\right)
\end{eqnarray}
\end{itemize}
\end{frame}
%================================================================================================================



\section{LSST Rubin science pipelines}
\begin{frame}{Rubin-LSST science pipeline}
\includegraphics[width=12cm, height=12cm]{figs/dm/DMpipelines1.png}
\end{frame}

\begin{frame}{Rubin-LSST science pipeline}
\begin{tabular}{cc}
\includegraphics[width=6cm, height=3.5cm]{figs/dm/DMpipelines11.png}
&
\includegraphics[width=6cm, height=8cm]{figs/dm/DMpipelines2.png}
\end{tabular}
\end{frame}

\begin{frame}{Rubin-LSST calibration plan}
\includegraphics[width=12cm, height=8cm]{figs/calib/LSSTCalibrationPlan.png}
\end{frame}


\section{Simulation of the photometric corrections}

\subsection{Atmospheric simulation and $S_b^{obs}$}
%====================================================================
\begin{frame}{Atmospheric simulation and $S_b^{obs}$}
\begin{tabular}{cc}
\includegraphics[width=5.9cm, height=3.5cm]{figs/atmsimu/atmsimvsairmass.png} & \includegraphics[width=5.9cm, height=3.5cm]{figs/atmsimu/trb_vs_airmass.png} \\
\includegraphics[width=5.9cm, height=3.5cm]{figs/atmsimu/trb_vs_VAOD.png} & \includegraphics[width=5.9cm, height=3.5cm]{figs/atmsimu/trb_vs_PWV.png}
\end{tabular}
\end{frame}
%====================================================================


\subsection{The zero's order of the photometric correction}
%====================================================================
\begin{frame}{The $0^{th}$ order of the photometric correction}
\begin{alertblock}{vs airmass, relative to the standard transmission}
\begin{equation}
\mathbb{I}^{obs}_{0}(b,z) - \mathbb{I}^{std}_{0}(b,z_{std})
\end{equation}
\end{alertblock}
\includegraphics[width=10cm, height=5cm]{figs/Integrals_II/Integrals_II0_vs_airmass.png}
\end{frame}
%====================================================================


%====================================================================
\begin{frame}{The $0^{th}$ order of the photometric correction}
\begin{alertblock}{vs aeorols or PWV, relative to the standard transmission}
\begin{equation}
\mathbb{I}^{obs}_{0}(b,z_{std}) - \mathbb{I}^{std}_{0}(b,z_{std})
\end{equation}
\end{alertblock}
\begin{tabular}{cc}
\includegraphics[width=5.9cm, height=4cm]{figs/Integrals_II/Integrals_II0_vs_VAOD.png} & \includegraphics[width=5.9cm, height=4cm]{figs/Integrals_II/Integrals_II0_vs_PWV.png}
\end{tabular}
\end{frame}
%====================================================================

\subsection{The 1st and 2nd orders Integral differences}
%====================================================================
\begin{frame}{The $1^{st}$\&$2^{nd}$ orders Integral differences}

\begin{tabular}{cc}
\includegraphics[width=5.9cm, height=4cm]{figs/Integrals_II/Integrals_II10_vs_airmass.png} & \includegraphics[width=5.9cm, height=4cm]{figs/Integrals_II/Integrals_II20_vs_airmass.png}
\end{tabular}
\end{frame}
%====================================================================


%==========================================================================================================================
\begin{frame}
\begin{center}
{\usebeamerfont{frametitle}

\LARGE \alert{End of part 1}}

\end{center}

\end{frame}
%==========================================================================================================================

 
\end{document}


